% Don't touch this %%%%%%%%%%%%%%%%%%%%%%%%%%%%%%%%%%%%%%%%%%%
\documentclass[11pt]{article}
\usepackage{fullpage}
\usepackage[left=1in,top=1in,right=1in,bottom=1in,headheight=3ex,headsep=3ex]{geometry}
\usepackage{graphicx}
\usepackage{float}

\newcommand{\blankline}{\quad\pagebreak[2]}
%%%%%%%%%%%%%%%%%%%%%%%%%%%%%%%%%%%%%%%%%%%%%%%%%%%%%%%%%%%%%%

% Modify Course title, instructor name, semester here %%%%%%%%

\title{CSCI 5300: Computing Security (3)}
\author{Calvin Deutschbein}
\date{2025 Summer, MS Computer Science Mini-Term I}

%%%%%%%%%%%%%%%%%%%%%%%%%%%%%%%%%%%%%%%%%%%%%%%%%%%%%%%%%%%%%%

% Don't touch this %%%%%%%%%%%%%%%%%%%%%%%%%%%%%%%%%%%%%%%%%%%
\usepackage[sc]{mathpazo}
\linespread{1.05} % Palatino needs more leading (space between lines)
\usepackage[T1]{fontenc}
\usepackage[mmddyyyy]{datetime}% http://ctan.org/pkg/datetime
\usepackage{advdate}% http://ctan.org/pkg/advdate
\newdateformat{syldate}{\twodigit{\THEMONTH}/\twodigit{\THEDAY}}
\newsavebox{\MONDAY}\savebox{\MONDAY}{Mon}% Mon
\newcommand{\week}[1]{%
%  \cleardate{mydate}% Clear date
% \newdate{mydate}{\the\day}{\the\month}{\the\year}% Store date
  \paragraph*{\kern-2ex\quad #1, \syldate{\today} - \AdvanceDate[4]\syldate{\today}:}% Set heading  \quad #1
%  \setbox1=\hbox{\shortdayofweekname{\getdateday{mydate}}{\getdatemonth{mydate}}{\getdateyear{mydate}}}%
  \ifdim\wd1=\wd\MONDAY
    \AdvanceDate[7]
  \else
    \AdvanceDate[7]
  \fi%
}
\usepackage{setspace}
\usepackage{multicol}
%\usepackage{indentfirst}
\usepackage{fancyhdr,lastpage}
\usepackage{url}
\pagestyle{fancy}
\usepackage{hyperref}
\usepackage{lastpage}
\usepackage{amsmath}
\usepackage{layout}
\usepackage{csquotes}
\setlength\parindent{0pt}

\lhead{}
\chead{}
%%%%%%%%%%%%%%%%%%%%%%%%%%%%%%%%%%%%%%%%%%%%%%%%%%%%%%%%%%%%%%

% Modify header here %%%%%%%%%%%%%%%%%%%%%%%%%%%%%%%%%%%%%%%%%
\rhead{\footnotesize Text in header}

%%%%%%%%%%%%%%%%%%%%%%%%%%%%%%%%%%%%%%%%%%%%%%%%%%%%%%%%%%%%%%
% Don't touch this %%%%%%%%%%%%%%%%%%%%%%%%%%%%%%%%%%%%%%%%%%%
\lfoot{}
\cfoot{\small \thepage/\pageref*{LastPage}}
\rfoot{}

\usepackage{array, xcolor}
\usepackage{color,hyperref}
\definecolor{clemsonorange}{HTML}{EA6A20}
\hypersetup{colorlinks,breaklinks,linkcolor=clemsonorange,urlcolor=clemsonorange,anchorcolor=clemsonorange,citecolor=black}

\begin{document}

\maketitle

\blankline

\begin{tabular*}{\textwidth}{@{\extracolsep{\fill}}lr}

%%%%%%%%%%%%%%%%%%%%%%%%%%%%%%%%%%%%%%%%%%%%%%%%%%%%%%%%%%%%%%

% Modify information %%%%%%%%%%%%%%%%%%%%%%%%%%%%%%%%%%%%%%%%%
E-mail: \href{mailto:cdeutschbein@lagrange.edu}{\tt\bf cdeutschbein@lagrange.edu} & Web: \href{https://cd-public.github.io/scicom/}{\tt\bf https://cd-public.github.io/scicom/}  \\

07/07/25 - 08/08/25 &  Asynchronous \\
\hline
\end{tabular*}

\vspace{5 mm}

% Zeroth Section %%%%%%%%%%%%%%%%%%%%%%%%%%%%%%%%%%%%%%%%%%%%

\section*{Modality and Credit Hour Compliance}

\subsection*{Residency}  

We will meet in person at LaGrange College from 8 AM to 6 PM on on 6/6-7/25
and from 8 AM to 12 Noon on 6/8/2025. These lectures will also be recorded and posted on
the course website. This is 22 of the 37.5 contact hours resulting in 3 credit hour course.

\subsection*{Asynchronity} 

I will upload a weekly lecture video to YouTube, with link posted to the course
website on GitHub pages, every Monday from 7/13/25 to 3/3/25. This is 15.5 of the 37.5 contact
hours resulting in a 3 credit hour course. Separately, I will be available asynchronous via email
for terminating and Discord for persistent communication.

\subsection*{Dates}

While instruction will be conducted  07/07/25 - 08/08/25, there are a few relevant dates outside of that range related to the academic calendar. Additional, students may begin working on the course at any time, with deadlines only falling in the  07/07/25 - 08/08/25 range.

\begin{itemize}
	\item Fri, June 20, 2025 Last Day of Add/Drop at 5:00pm
	\item Wed, July 23, 2025 Last Day to Withdraw with a “W” at 5:00pm
	\item Fri, August 08, 2025 Last Day
\end{itemize}

\subsection*{Deliverables} 

Students will be responsible for submitting a scientific report using (1) scientific computing taught during the residency and (2) a scientific publishing technology taught asynchronously. Assignments will be due weekly from 6/16/25 to 7/25/25. Assignments will be due every Monday at 12 midnight AOE following the residency.  I
will target 12 hours of effort each across weekly problem sets resulting in a 2-3 hours homework
per contact hour ratio in accordance with my understanding of credit hour policy.

% First Section %%%%%%%%%%%%%%%%%%%%%%%%%%%%%%%%%%%%%%%%%%%%

\section*{Course Description}

A study of high-performance computing for advanced scientific research on modern processors. Topics include high-performance computing techniques, floating point properties, and advanced numerical methods.

% Second Section %%%%%%%%%%%%%%%%%%%%%%%%%%%%%%%%%%%%%%%%%%%

\section*{Course Materials}

\begin{itemize}
\item Course materials at \href{https://cd-public.github.io/scicom/}{\tt\bf https://cd-public.github.io/scicom}
\item Optional Textbook: The C Programming Language, Brian Kernighan\& Dennis Ritchie
\item A Graduate Course in Applied Cryptography Dan Boneh \& Victor Shoup
\item Supplemental Material: \href{https://cd-public.github.io/courses/old/c89s25/index.html/}{\tt\bf The 4 credit hour version of this course.}
\end{itemize}

% Third Section %%%%%%%%%%%%%%%%%%%%%%%%%%%%%%%%%%%%%%%%%%%

\section*{Prerequisite}
B.S. Computer Science or equivalent.

% Fourth Section %%%%%%%%%%%%%%%%%%%%%%%%%%%%%%%%%%%%%%%%%%%

\section*{Course Objectives}
\textbf{\textit{LaGrange College Student Learning Outcomes (LC SLO):}}

\begin{enumerate}
\item Students will demonstrate \underline{creativity} by approaching complex problems with innovation and from
\item Students will demonstrate \underline{critical thinking} by acquiring, interpreting, synthesizing, and evaluating
information to reason out conclusions appropriately.
\item Students will demonstrate proficiency in \underline{communication} skills that are applicable to any field of
study.
\end{enumerate}

\textbf{\textit{Student Learning Outcomes (SLO) for CISC 5300}}

\textit{All learning objectives pursuant to LC SLO \{1,2,3\} and to be assessed by homework assignments.}

\begin{enumerate}
    \item \textbf{Foundational Programming Concepts and Memory Management}
        \begin{itemize}
            \item \textbf{SLO 1:} Explain and apply concepts of pointers and memory safety in programming.
            \item \textbf{SLO 2:} Design and implement recursive algorithms for problem-solving.
            \item \textbf{SLO 3:} Develop programs using the C89 standard and manage the build process using Makefiles and shell scripts.
        \end{itemize}
    \item \textbf{Data Structures and Modern Development Practices}
        \begin{itemize}
            \item \textbf{SLO 4:} Implement and analyze the fundamental operations of various data structures, including lists, stacks, heaps, maps, and trees.
            \item \textbf{SLO 5:} Utilize automated testing to ensure code and/or software quality.
        \end{itemize}
    \item \textbf{Cryptography and Mathematical Foundations}
        \begin{itemize}
            \item \textbf{SLO 6:} Explain the principles behind fundamental cryptographic algorithms such as RSA and SHA.
            \item \textbf{SLO 7:} Apply concepts from number theory, set theory, and graph theory to solve problems in computer science.
        \end{itemize}
    \item \textbf{Cybersecurity Principles}
        \begin{itemize}
            \item \textbf{SLO 8:} Identify and analyze common cybersecurity threat models and explain the principles of Confidentiality, Integrity, and Availability (CIA).
        \end{itemize}
\end{enumerate}

% 4.5th Section %%%%%%%%%%%%%%%%%%%%%%%%%%%%%%%%%%%%%%%%%%%

\subsection*{Assignments and Assessment}

\begin{itemize}
    \item Students begin the term with a default grade of "A".
    \item This course utilizes objective, automated, binary grading on five weekly assignments.
        \begin{itemize}
            \item  Every assignments comes furnished with an autograder.
            \begin{itemize}
                \item  It is provided to students with the assignment specification.
                \item  It is documented and preserved under version control.
                \item There are special allowances for the fourth assignment.
            \end{itemize}
            \item  Students submit C language source code that accomplishes the goals of the autograder.
            \begin{itemize}
                \item  Students may use any resources but,
                \item  Students are responsible for knowing the content of the course.
                \item  We accept the autograder as ground truth.
            \end{itemize}
            \item  Automated assessment will score student work.
            \begin{itemize}
                \item  Late assignments earn zero points.
                \item  Assignments that do not fulfill the autograder requirements earn zero points.
                \item  Students lose one letter grade per failed assignment.
            \end{itemize}
        \end{itemize}
\end{itemize}

% Fifth Section %%%%%%%%%%%%%%%%%%%%%%%%%%%%%%%%%%%%%%%%%%%

\section*{College Policies}

\subsection*{ADA Statement}

In compliance with Section 504 of the Rehabilitation Act and the Americans with Disabilities Act,
LaGrange College will provide reasonable accommodation of all medically documented disabilities. If you
have a disability and would like the College to provide reasonable accommodations of the disability during
this course, please notify Ms. Lindsay Shaughnessy, Director of the Panther Academic Center for
Excellence (PACE) and Coordinator of Accessibility Services at \href{mailto:accessability@lagrange.edu}{accessability@lagrange.edu} or 706-880-8652. PACE is located in the Moshell Learning Center \&
Tutoring Lab in the Lewis Library.


\subsection*{Academic Support}

Academic Support
Academic support at LaGrange is provided through Panther Academic Center for Excellence (PACE), the
Writing Center, and the advising deans. PACE provides peer tutoring, testing services, accessibility
services, and other academic support as needed. For more information about PACE, please contact Mr.
Steve Kenner (skenner@lagrange.edu). The Writing Center gives all writers a space to explore the potential of their ideas via peer review. For information about the Writing Center, contact Dr. Justin Thurman
(jthurman@lagrange.edu).

\subsection*{Academic Integrity Policy}

Each student is bound by the LaGrange College Honor Code which is stated as follows:

\begin{displayquote}
As a member of the student body of LaGrange College, I confirm my
commitment to the ideals of civility, diversity, service, and excellence.
Recognizing the significance of personal integrity in establishing these ideals
within our community, I pledge that I will not lie, cheat, steal, nor tolerate these
unethical behaviors in others.
\end{displayquote}

The full text of the LaGrange College Honor Code along with policies and procedures in cases of academic
dishonesty can be found at \url{http://www.lagrange.edu/resources/pdf/honorcode12-13.pdf}.

\subsection*{Academic Integrity Policy}

Email and LaGrange college accounts will be used in accordance with the following student handbook
statement:

\begin{displayquote}
“Students are expected to treat their campus [e­mail] accounts as a business account.
Faculty and administrators rely on these accounts to disseminate important information regarding
College protocol and events. Therefore, students are responsible for any College information sent
out over campus e­mail.”
\end{displayquote}

Consequently, personal email addresses will not be used for instructor/student email contact except in event a service interruption. The preferred
method of contact will instead be by the official campus email. I target a 24 hour maximum response time on school days and 48 hours maximum response time on all emails while the course is active.

As an adjunct, my LaGrange email may not persistent indefinitely. I maintain a persistent professional email at \href{calvindeu@gmail.com}{mailto:calvindeu@gmail.com} which can also be used in event service interrupts to the campus network or for professional references after the conclusion of the course.

\subsection*{Netiquette}

When leaving comments or asking questions in the forums of an online course, one is reminded to observe
a few rules of internet etiquette:

\begin{itemize}
\item All caps locks and/or multiple exclamation points typically imply anger. You should not use such
emphases unless it accomplishes a learning objective.
\item Vulgarity, rudeness, and/or disrespect are complete unacceptable and will not be tolerated.
\item Emoticons (such as ‘:)’ for a ‘smiley face’) are fine for use in relaxed submissions (forum threads
and posts).
\item In general, do your best to use proper spelling, grammar, and punctuation. Writing correctly
works to ensure that your meaning is conveyed.
\end{itemize}

\subsubsection*{Technology Requirements}

To achieve the learning objectives of this course, the following development environment is required, and available free of charge and open-source on any computing platform:

\begin{itemize}
\item A Linux distribution (I recommend Ubuntu) with the following utilities:
\begin{itemize}
\item The "gcc" C compiler
\item The "vim" modal text editor.
\item The "git" version control system.
\item The "podman" container management tool.
\end{itemize}
\end{itemize}

\subsubsection*{Technical Support}

I will independently offer technology support for the technology stack used to support this course. Contact me directly unless you have technical issues arising within LaGrange.edu realms, in which case you should reach out via email to \href{support@lagrange.edu}{mailto:support@lagrange.edu} or call 706.880.8053.

Precise technical writing is a core learning objective (LC SLO 3) for this course, and should be modeled in all technical support interactions.

\subsubsection*{Agreement by Continued Enrollment}

By remaining enrolled in the course, each student agrees to the terms of the syllabus as a binding contract between the student, the instructor, and LaGrange College.

\subsubsection*{Note on attending asynchronous attendance:}

I am confident I have formulated the assessment tools such that attendance or non-attendance by individuals, as measured by viewing of asynchronous lectures, will be obvious to me as an instructor. As such, I have folded my attendance considerations into the assessment formulation.

It is trivial as an instructor to assess the level of engagement with asynchronous learning resources, and you should regard it as more, not less, clear what a student's level of participation is for asynchronous instruction.

\subsubsection*{J1 Retention}

J1 Retention is a tool used by LaGrange College faculty and staff to promote student academic and extra-curricular success. Expect us to use it to report information about attendance, engagement, or academic performance on specific assignments throughout the semester. Faculty and staff, such as coaches, the PACE Director, and advising deans, will be notified of relevant concerns. Based on provided information, you may receive automated messages from J1 Retention, referrals to the Tutoring or Writing Centers, or request to meet with your advising dean or the PACE director if there are indicators that you might benefit from additional support. 


\end{document}
